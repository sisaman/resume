%%%%%%%%%%%%%%%%%%%%%%%%%%%%%%%%%%%%%%%%%%%%%%%%%%%%%%%%%%%%%%%%%%%%%%%%
%%%%%%%%%%%%%%%%%%%%%% Simple LaTeX CV Template %%%%%%%%%%%%%%%%%%%%%%%%
%%%%%%%%%%%%%%%%%%%%%%%%%%%%%%%%%%%%%%%%%%%%%%%%%%%%%%%%%%%%%%%%%%%%%%%%

%%%%%%%%%%%%%%%%%%%%%%%%%%%%%%%%%%%%%%%%%%%%%%%%%%%%%%%%%%%%%%%%%%%%%%%%
%% NOTE: If you find that it says                                     %%
%%                                                                    %%
%%                           1 of ??                                  %%
%%                                                                    %%
%% at the bottom of your first page, this means that the AUX file     %%
%% was not available when you ran LaTeX on this source. Simply RERUN  %%
%% LaTeX to get the ``??'' replaced with the number of the last page  %%
%% of the document. The AUX file will be generated on the first run   %%
%% of LaTeX and used on the second run to fill in all of the          %%
%% references.                                                        %%
%%%%%%%%%%%%%%%%%%%%%%%%%%%%%%%%%%%%%%%%%%%%%%%%%%%%%%%%%%%%%%%%%%%%%%%%

%%%%%%%%%%%%%%%%%%%%%%%%%%%% Document Setup %%%%%%%%%%%%%%%%%%%%%%%%%%%%

% Don't like 10pt? Try 11pt or 12pt
\documentclass[8pt]{article}
\RequirePackage[T1]{fontenc}

% LaTeX will typeset using Computer Modern Roman, which a lot of
% non-mathematicians and non-engineers won't like. Also, a few PDF
% viewers may not render CMR very well. Instead, Times New Roman can
% be used. That's what this package does.
\usepackage{times}
\usepackage{tabu}
\usepackage{fontawesome}
% The automated optical recognition software used to digitize resume
% information works best with fonts that do not have serifs. This
% command uses a sans serif font throughout. Uncomment both lines (or at
% least the second) to restore a Roman font (i.e., a font with serifs).
% (NOTE: This requires the times package above)
%\renewcommand{\familydefault}{\sfdefault}

% This is a helpful package that puts math inside length specifications
\usepackage{calc}

% This package helps LaTeX auto-hyphenate hyphenated words if you use
% special hyphens. For example, bio\-/mimicry will properly hyphenate
% ``mimicry'' if necessary.
\usepackage[shortcuts]{extdash}

% Layout: Puts the section titles on left side of page
\reversemarginpar

%
%         PAPER SIZE, PAGE NUMBER, AND DOCUMENT LAYOUT NOTES:
%
% The next \usepackage line changes the layout for CV style section
% headings as marginal notes. It also sets up the paper size as either
% letter or A4. By default, letter was used. If A4 paper is desired,
% comment out the letterpaper lines and uncomment the a4paper lines.
%
% As you can see, the margin widths and section title widths can be
% easily adjusted.
%
% ALSO: Notice that the includefoot option can be commented OUT in order
% to put the PAGE NUMBER *IN* the bottom margin. This will make the
% effective text area larger.
%
% IF YOU WISH TO REMOVE THE ``of LASTPAGE'' next to each page number,
% see the note about the +LP and -LP lines below. Comment out the +LP
% and uncomment the -LP.
%
% IF YOU WISH TO REMOVE PAGE NUMBERS, be sure that the includefoot line
% is uncommented and ALSO uncomment the \pagestyle{empty} a few lines
% below.
%

%% Use these lines for letter-sized paper
\usepackage[paper=a4paper,
            %includefoot, % Uncomment to put page number above margin
            marginparwidth=1in,     % Length of section titles
            marginparsep=.05in,       % Space between titles and text
            margin=0.5in,               % 1 inch margins
            top=1in,
            includemp
            ]{geometry}

%% Use these lines for A4-sized paper
%\usepackage[paper=a4paper,
%            %includefoot, % Uncomment to put page number above margin
%            marginparwidth=30.5mm,    % Length of section titlesa
%            marginparsep=1.5mm,       % Space between titles and text
%            margin=25mm,              % 25mm margins
%            includemp]{geometry}

%% More layout: Get rid of indenting throughout entire document
\setlength{\parindent}{0in}

% Provides special list environments and macros to create new ones
\usepackage[shortlabels]{enumitem}

\usepackage[nodayofweek]{datetime}
\newdateformat{mydate}{\twodigit{\THEDAY}{ }\shortmonthname[\THEMONTH], \THEYEAR}


% Simpler bibsections for CV sections
% (thanks to natbib for inspiration)
%
% * For lists of references with hanging indents and no numbers:
%
%   \begin{bibsection}
%       \item ...
%   \end{bibsection}
%
% * For numbered lists of references (with hanging indents):
%
%   \begin{bibenum}
%       \item ...
%   \end{bibenum}
%
%   Note that bibenum numbers continuously throughout. To reset the
%   counter, use
%
%   \restartlist{bibenum}
%
%   at the place where you want the numbering to reset.

\makeatletter
\newlength{\bibhang}
\setlength{\bibhang}{1em}
\newlength{\bibsep}
 {\@listi \global\bibsep\itemsep \global\advance\bibsep by\parsep}
\newlist{bibsection}{itemize}{3}
\setlist[bibsection]{label=,leftmargin=\bibhang,%
        itemindent=-\bibhang,
        itemsep=\bibsep,parsep=\z@,partopsep=0pt,
        topsep=0pt}
\newlist{bibenum}{enumerate}{3}
\setlist[bibenum]{label=[\arabic*],resume,leftmargin={\bibhang+\widthof{[999]}},%
%        itemindent=-\bibhang,
%        itemsep=\bibsep,parsep=\z@,partopsep=0pt,
        topsep=0pt
      }
\let\oldendbibenum\endbibenum
\def\endbibenum{\oldendbibenum\vspace{-.6\baselineskip}}
\let\oldendbibsection\endbibsection
\def\endbibsection{\oldendbibsection\vspace{-.6\baselineskip}}
\makeatother

%% Reference the last page in the page number
%
% NOTE: comment the +LP line and uncomment the -LP line to have page
%       numbers without the ``of ##'' last page reference)
%
% NOTE: uncomment the \pagestyle{empty} line to get rid of all page
%       numbers (make sure includefoot is commented out above)
%
\usepackage{fancyhdr,lastpage}

\newlength{\myoddoffset}
\setlength{\myoddoffset}{\marginparwidth + \marginparsep}

\pagestyle{fancy}
%\pagestyle{empty}      % Uncomment this to get rid of page numbers

\fancyheadoffset[leh,roh]{\marginparsep}
\fancyheadoffset[loh,reh]{\myoddoffset}

\fancyhf{}\renewcommand{\headrulewidth}{1pt}
\fancyhead[L]{\large\textbf{Sina Sajadmanesh}}

\fancyfootoffset{\marginparsep+\marginparwidth}
\newlength{\footpageshift}
\setlength{\footpageshift}
          {0.5\textwidth+0.5\marginparsep+0.5\marginparwidth-2in}
 
\fancyfoot[C]{}         
\fancyfoot[L]{
	\hspace{\footpageshift}%
	\parbox{4in}{\, \hfill %
		\arabic{page} of \protect\pageref*{LastPage} % +LP
		\hfill \,}
}

\fancyhead[R]{
	Last update: \mydate\today
}
% Finally, give us PDF bookmarks
\usepackage{color,hyperref}
\definecolor{darkblue}{rgb}{0.0,0.0,0.3}
\hypersetup{colorlinks,breaklinks,
            linkcolor=darkblue,urlcolor=darkblue,
            anchorcolor=darkblue,citecolor=darkblue}
\newcommand{\MYhref}[3][blue]{\href{#2}{\color{#1}{#3}}}%
%%%%%%%%%%%%%%%%%%%%%%%% End Document Setup %%%%%%%%%%%%%%%%%%%%%%%%%%%%


%%%%%%%%%%%%%%%%%%%%%%%%%%% Helper Commands %%%%%%%%%%%%%%%%%%%%%%%%%%%%

%%% HEADING AT TOP OF CURRICULUM VITAE

% The title (name) with a horizontal rule under it
% (optional argument typesets an object right-justified across from name
%  as well)
%
% Usage: \makeheading{name}
%        OR
%        \makeheading[right_object]{name}
%
% Place at top of document. It should be the first thing.
% If ``right_object'' is provided in the square-braced optional
% argument, it will be right justified on the same line as ``name'' at
% the top of the CV. For example:
%
%       \makeheading[\emph{Curriculum vitae}]{Your Name}
%
% will put an emphasized ``Curriculum vitae'' at the top of the document
% as a title. Likewise, a picture could be included:
%
%   \makeheading[{\includegraphics[height=1.5in]{my_picture}}]{Your Name}
%
% the picture will be flush right across from the name. For this example
% to work, make sure the extra set of curly braces is included. Also
% makes ure that \usepackage{graphicx} is somewhere in the preamble.
\newcommand{\makeheading}[2][]%
        {\hspace*{-\marginparsep minus \marginparwidth}%
         \begin{minipage}[t]{\textwidth+\marginparwidth+\marginparsep}%
             {\large \bfseries #2 \hfill #1}\\[-0.15\baselineskip]%
                 \rule{\columnwidth}{1pt}%
         \end{minipage}}

%%% SECTION HEADINGS

% The section headings. Flush left in small caps down pseudo-margin.
%
% Usage: \section{section name}
\renewcommand{\section}[1]{\pagebreak[3]%
    \vspace{1.3\baselineskip}%
    \phantomsection\addcontentsline{toc}{section}{#1}%
    \noindent\llap{\scshape\smash{\parbox[t]{\marginparwidth}{\hyphenpenalty=10000\raggedright #1}}}%
    \vspace{-\baselineskip}\par}

%%% LISTS

% This macro alters a list by removing some of the space that follows the list
% (is used by lists below)
\newcommand*\fixendlist[1]{%
    \expandafter\let\csname preFixEndListend#1\expandafter\endcsname\csname end#1\endcsname
    \expandafter\def\csname end#1\endcsname{\csname preFixEndListend#1\endcsname\vspace{-0.6\baselineskip}}}

% These macros help ensure that items in outer-type lists do not get
% separated from the next line by a page break
% (they are used by lists below)
\let\originalItem\item
\newcommand*\fixouterlist[1]{%
    \expandafter\let\csname preFixOuterList#1\expandafter\endcsname\csname #1\endcsname
    \expandafter\def\csname #1\endcsname{\let\oldItem\item\def\item{\pagebreak[2]\oldItem}\csname preFixOuterList#1\endcsname}
    \expandafter\let\csname preFixOuterListend#1\expandafter\endcsname\csname end#1\endcsname
    \expandafter\def\csname end#1\endcsname{\let\item\oldItem\csname preFixOuterListend#1\endcsname}}
\newcommand*\fixinnerlist[1]{%
    \expandafter\let\csname preFixInnerList#1\expandafter\endcsname\csname #1\endcsname
    \expandafter\def\csname #1\endcsname{\let\oldItem\item\let\item\originalItem\csname preFixInnerList#1\endcsname}
    \expandafter\let\csname preFixInnerListend#1\expandafter\endcsname\csname end#1\endcsname
    \expandafter\def\csname end#1\endcsname{\csname preFixInnerListend#1\endcsname\let\item\oldItem}}

% An itemize-style list with lots of space between items
%
% Usage:
%   \begin{outerlist}
%       \item ...    % (or \item[] for no bullet)
%   \end{outerlist}
\newlist{outerlist}{itemize}{3}
    \setlist[outerlist]{label=\enskip\textbullet,leftmargin=*}
    \fixendlist{outerlist}
    \fixouterlist{outerlist}

% An environment IDENTICAL to outerlist that has better pre-list spacing
% when used as the first thing in a \section
%
% Usage:
%   \begin{lonelist}
%       \item ...    % (or \item[] for no bullet)
%   \end{lonelist}
\newlist{lonelist}{itemize}{3}
    \setlist[lonelist]{label=\enskip\textbullet,leftmargin=*,partopsep=0pt,topsep=0pt}
    \fixendlist{lonelist}
    \fixouterlist{lonelist}

% An itemize-style list with little space between items
%
% Usage:
%   \begin{innerlist}
%       \item ...    % (or \item[] for no bullet)
%   \end{innerlist}
\newlist{innerlist}{itemize}{3}
    \setlist[innerlist]{label=\enskip\textbullet,leftmargin=*,parsep=0pt,itemsep=0pt,topsep=0pt,partopsep=0pt}
    \fixinnerlist{innerlist}

% An environment IDENTICAL to innerlist that has better pre-list spacing
% when used as the first thing in a \section
%
% Usage:
%   \begin{loneinnerlist}
%       \item ...    % (or \item[] for no bullet)
%   \end{loneinnerlist}
\newlist{loneinnerlist}{itemize}{3}
    \setlist[loneinnerlist]{label=\enskip\textbullet,leftmargin=*,parsep=0pt,itemsep=0pt,topsep=0pt,partopsep=0pt}
    \fixendlist{loneinnerlist}
    \fixinnerlist{loneinnerlist}

%%% EXTRA SPACE

% To add some paragraph space between lines.
% This also tells LaTeX to preferably break a page on one of these gaps
% if there is a needed pagebreak nearby.
\newcommand{\blankline}{\quad\pagebreak[3]}
\newcommand{\halfblankline}{\quad\vspace{-0.5\baselineskip}\pagebreak[3]}

%%% FORMATTING MACROS

% Provides a linked \doi{#1} that links doi:#1 to http://dx.doi.org/#1
\usepackage{doi}
% To change the text before the DOI, adjust this command
%\renewcommand\doitext{doi:}

% Provides a linked \url{#1} that doesn't require escape characters
\usepackage{url}

% You can adjust the style \url{} uses here:
% (options are: same, rm, sf, tt; defaults to tt)
\urlstyle{same}

% For \email{ADDRESS}, links ADDRESS to the url mailto:ADDRESS
% (uncomment to typeset the e\-/mail address in typewriter font;
%  otherwise, will be typeset in the \urlstyle above)
%\DeclareUrlCommand\emaillink{\urlstyle{tt}}
\providecommand*\emaillink[1]{\nolinkurl{#1}}
\providecommand*\email[1]{\href{mailto:#1}{\emaillink{#1}}}

\providecommand\BibTeX{{B\kern-.05em{\sc i\kern-.025em b}\kern-.08em \TeX}}
\providecommand\Matlab{\textsc{Matlab}}

% Custom hyphenation rules for words that LaTeX has trouble with
\hyphenation{bio-mim-ic-ry bio-in-spi-ra-tion re-us-a-ble pro-vid-er Media-Wiki}

%%%%%%%%%%%%%%%%%%%%%%%% End Helper Commands %%%%%%%%%%%%%%%%%%%%%%%%%%%
\def\longversion{1}  % set to 0 for short version
%%%%%%%%%%%%%%%%%%%%%%%%% Begin CV Document %%%%%%%%%%%%%%%%%%%%%%%%%%%%

\begin{document}
%\makeheading{Sina Sajadmanesh}

\section{Contact Information}

% NOTE: Mind where the & separators and \\ breaks are in the following
%       table. Table is one row made up of three parboxes. The left
%       parbox has address info, the middle parbox has a vertical bar,
%       and the right parbox has phone and electronic contact
%       information.
%
% MACROS: \rcollength is the width of the right column of the table
%             (adjust it to your liking; default is 1.85in).
%         \spacewidth is width of area between left and right boxes.
%
%\newlength{\rcollength}\setlength{\rcollength}{0in}%
\newlength{\spacewidth}\setlength{\spacewidth}{20pt}
%
%\begin{tabular}[t]{@{}p{\textwidth-\rcollength-\spacewidth}@{}p{\spacewidth}@{}p{\rcollength}}%
\begingroup
\setlength{\tabcolsep}{0pt} % Default value: 6pt
\begin{tabu} to \textwidth {X[l]X[r]}
	EPFL STI IEM LIDIAP & (+41) 27-721-77-58~\faPhone\\
	ELD 241 (Bâtiment ELD) & \hfill{\href{mailto:sina.sajadmanesh@epfl.ch}{sina.sajadmanesh@epfl.ch}~\faEnvelope} \\ 
	Station 11, CH-1015 Lausanne & \hfill{\href{https://sajadmanesh.com}{https://sajadmanesh.com}~\faHome} \\
\end{tabu}
\endgroup
%
%Rue Marconi 19 \hfill (+41) 27-721-77-58~\faPhone\\
%1920 Martigny \hfill{\href{mailto:sajadmanesh@idiap.ch}{sajadmanesh@idiap.ch}~\faEnvelope} \\ 
%Switzerland \hfill{\href{https://sajadmanesh.com}{https://sajadmanesh.com}~\faHome}

%\hfill{GitHub: \href{http://www.github.com/pauljwright}{www.github.com/pauljwright}} }


%\end{tabular}

%%
%% In modern CV's, it seems like ``Objective'' is frowned upon. Instead,
%% incorporate it into a well-constructed cover letter. The ``More
%% information'' can go at the end of the CV, but it should not distract
%% from the section giving references available to contact.
%%
%
% \section{Objective}
%
% Placement in an academic position (i.e., faculty, postdoctoral, or
% research scientist) that allows for advanced research in distributed
% complex adaptive systems (i.e., modeling, analysis, design, and
% verification) with a particular focus on the control of engineered
% agents (e.g., for communications, control, software, electronics, and
% sustainability) and the analysis of biological phenomena (e.g.,
% self-organization, ecological rationality)
% \begin{innerlist}
% \item More information and auxiliary documents can be found at\\\url{http://www.tedpavlic.com/facjobsearch/}
% \end{innerlist}

\if\longversion1
\section{Research Interests}

% My research interests lie at the intersection of privacy, deep learning, and graph analysis. More specifically, I use privacy enhancing technologies, such as differential privacy and federated learning, with graph representation learning algorithms, including graph neural networks, to make them more private, secure, and robust for real-world applications.

Differential Privacy, Trustworthy Machine Learning, Federated Learning, Graph Representation Learning
\fi


%%%% EDUCATION %%%%
\section{Education}

\href{https://www.epfl.ch/en/home/}{\textbf{École Polytechnique Fédérale de Lausanne (EPFL)}}, Lausanne, Switzerland, {May 2019 -- August 2023}
\begin{innerlist}
\item[] Ph.D. in Electrical Engineering \quad GPA: 5.7 / 6
        \begin{innerlist}
        \item[] \textbf{Thesis:} \emph{Trustworthy Machine Learning on Graphs}
        % \item[] \textbf{Adviser:} Prof.~Daniel Gatica-Perez
        \if\longversion1
        \item[] \textbf{Relevant Courses:} Artificial Neural Networks (Deep Reinforcement Learning), Deep Learning for Natural Language Processing, Advanced Topics in Machine Learning
        \fi
        \end{innerlist}

\end{innerlist}

\halfblankline

\href{http://www.en.sharif.edu/}{\textbf{Sharif University of Technology}}, Tehran, Iran, {Sep 2014 -- Sep 2016}
\begin{innerlist}
	\item[] M.Sc. in Information Technology Engineering \quad  GPA: 18.1 / 20
	\begin{innerlist}
		\item[] \textbf{Thesis:} \emph{Link Prediction in Heterogeneous Multi-Layer Social Networks}
		% \item[] \textbf{Adviser:} Prof.~Hamid R. Rabiee
		\if\longversion1
		\item[] \textbf{Relevant Courses:} Machine Learning, Complex Dynamical Networks, Performance Modeling of Computer Systems, Advanced Network Security, Database Security and Privacy
		\fi
	\end{innerlist}
	
\end{innerlist}

\halfblankline

\href{http://ui.ac.ir/EN}{\textbf{University of Isfahan}}, Esfahan, Iran, {Sep 2009 -- Feb 2014}
\begin{innerlist}
	\item[] B.Sc. in Computer Software Engineering \quad  GPA: 16.19 / 20 (Last four semesters: 17.4 / 20)
	\begin{innerlist}
		\item[] \textbf{Project:} \emph{Design and Implementation of an Android App for Voice Control of Household Devices}
		% \item[] \textbf{Adviser:} Prof.~Ahamd R. Naghsh-Nilchi
		\if\longversion1
		\item[] \textbf{Relevant Courses:} Data Structures, Algorithms, Probability and Statistics, Artificial Intelligence, Information Retrieval, Software Engineering, Databases, Operating Systems, Computer Networks
		\fi
	\end{innerlist}
	
\end{innerlist}

\section{Research Experience}

\textbf{Visiting PhD Student}, {Feb 2023 -- April 2023}
\begin{innerlist}
    \item[] \href{https://www.turing.ac.uk/research/research-programmes/artificial-intelligence-ai/safe-and-ethical}{Safe and Ethical AI Programme}, \textbf{\href{https://www.turing.ac.uk/}{The Alan Turing Institute}}, London, UK
    \if\longversion1
    \begin{innerlist}
    	\item Working on trustworthy machine learning on graphs, aiming to address both privacy concerns and robustness issues of graph representation learning algorithms.
    \end{innerlist}
  	\fi
\end{innerlist}

\halfblankline


\textbf{Research Assistant}, {May 2019 -- present}
\begin{innerlist}
    \item[] \href{https://www.idiap.ch/en/scientific-research/social-computing/index_html}{Social Computing Group}, \textbf{\href{https://idiap.ch}{Idiap Research Institute}}, Martigny, Switzerland
    \if\longversion1
    \begin{innerlist}
    	\item Developing privacy-preserving graph neural network models using differential privacy to reduce the privacy risks of using graph representation learning algorithms in real applications.
    \end{innerlist}
  	\fi
\end{innerlist}

\halfblankline

\textbf{Research Intern}, {March 2022 -- May 2022}
\begin{innerlist}
    \item[] \textbf{\href{https://brave.com/}{Brave Software}}, San Francisco, CA, USA (Remote)
    \if\longversion1
    \begin{innerlist}
    	\item Worked on federated reinforcement learning algorithms to build privacy-preserving recommendation systems for Brave's ads and news recommendation.
    \end{innerlist}
  	\fi
\end{innerlist}

\halfblankline


\textbf{Research Assistant}, {Nov 2014 -- May 2019}
\begin{innerlist}
	\item[] %\href{http://dml.ce.sharif.edu/}
	{Data Science and Machine Learning Lab}, \href{http://www.en.sharif.edu/}{\textbf{Sharif University of Technology}}, Tehran, Iran
	\if\longversion1
	\begin{innerlist}
		\item \underline{Privacy-Preserving Deep Learning:} Worked on a hybrid mobile-server learning architecture based on Siamese fine-tuning and split learning to make non-private pre-trained deep learning models privacy-preserving at the inference stage.
		\item \underline{Web Data Science:} Analyzed a large-scale collection of recipes published on the web and their content, aiming to understand cuisines and culinary habits around the world.
		\item \underline{Social and Information Networks:} Developed time-aware link prediction algorithms over heterogeneous social networks using recurrent neural networks and non-parametric machine learning.
	\end{innerlist}
	\fi
\end{innerlist}

\section{Teaching Experience}

\textbf{Lecturer}, {November 2022}
\begin{innerlist}
	\item[] \href{https://www.i-aida.org}{\textbf{International AI Doctoral Academy (AIDA)}}, Online
	\begin{innerlist}
		\item[] \textbf{Course:} An Introduction to Trustworthy Machine Learning
		\item[] \textbf{Website:} \href{https://www.i-aida.org/course/an-introduction-to-trustworthy-machine-learning/}{https://www.i-aida.org/course/an-introduction-to-trustworthy-machine-learning/}
	\end{innerlist}
\end{innerlist}

\halfblankline


\textbf{Lecturer}, {Fall 2017}
\begin{innerlist}
	\item[] \href{http://ce.sharif.edu/}{Department of Computer Engineering}, \href{http://www.en.sharif.edu/}{\textbf{Sharif University of Technology}}, Tehran, Iran
	\begin{innerlist}
		\item[] \textbf{Course:} Fundamentals of Programming (Python)
		\item[] \textbf{Website:} \href{http://ce.sharif.edu/courses/96-97/1/ce153-12/}{http://ce.sharif.edu/courses/96-97/1/ce153-12/}
	\end{innerlist}
\end{innerlist}

\halfblankline

\textbf{Teaching Assistant}
\begin{innerlist}
	\item[] \href{http://www.en.sharif.edu/}{\textbf{EPFL}}
	\begin{innerlist}
		\item Computational Social Media (Spring 2021, 2022, and 2023)
	\end{innerlist}
	\item[] \href{http://www.en.sharif.edu/}{\textbf{Sharif University of Technology}}
	\begin{innerlist}
		\item Artificial Intelligence (Spring 2017),
		Advanced Topics in Artificial Intelligence - Statistical Learning Theory (Spring 2016),
		Engineering Probability and Statistics (Spring 2016)
	\end{innerlist}
%	\halfblankline
	\href{http://ui.ac.ir/EN}{\textbf{University of Isfahan}}
	\begin{innerlist}
		\item Artificial Intelligence (Fall 2013),
		Advanced Computer Programming 2 - JavaFx and Android  (Fall 2012),
		Computer Programming - Java (Fall 2011),
		Computer Programming - C++ (Fall 2010),
	\end{innerlist}
\end{innerlist}

%\newpage
%\makeheading{Sina Sajadmanesh} 

\section{Industrial Experience}

\textbf{Big-Data Engineer}, {Sep 2018 -- May 2019}
\begin{innerlist}
	\item[] \href{http://ictic.sharif.ir}{Sharif ICT Innovation Center}, Tehran, Iran
	\if\longversion1
	\begin{innerlist}
		\item Responsible for building a native big-data processing platform using state-of-the-art technologies, such as Spark, Cassandra, JanusGraph, Elasticsearch, etc.
	\end{innerlist}
	\fi
\end{innerlist}


% \halfblankline

% \textbf{Software Engineering Intern}, {Summer 2012}
% \begin{innerlist}
% 	\item[] Amin Computer Co., Esfahan, Iran
% 	\if\longversion1
% 	\begin{innerlist}
% 		\item Responsible for designing and developing an Android application for company's web-based human resource management system.
% 	\end{innerlist}
% 	\fi
% \end{innerlist}

 
%
% % Add a little space to nudge next ``Ref'd Journal Publications'' marginpar
% % down to make room for tall ``Submitted Journal Publications''
% % marginpar. If there are enough submitted journal publications, this
% % space will not be needed (and should be removed).
% \vspace{0.1in}

\section{Publications}

\begin{bibenum}
	\item{} \textbf{Sina Sajadmanesh} and Daniel Gatica-Perez\\
	\href{https://arxiv.org/abs/2304.08928}{\textbf{ProGAP: Progressive Graph Neural Networks with Differential Privacy Guarantees}}\\
	\textit{Technical Report, ArXiv e-prints}, Apr 2023

	\item{} \textbf{Sina Sajadmanesh}, Ali Shahin Shamsabadi, Aurélien Bellet, and Daniel Gatica-Perez\\
	\href{https://arxiv.org/abs/2203.00949}{\textbf{GAP: Differentially Private Graph Neural Networks with Aggregation Perturbation}}\\
	\textit{USENIX Security Symposium (USENIX Security 23)}, Aug 2023

	\item{} \textbf{Sina Sajadmanesh} and Daniel Gatica-Perez\\
	\href{https://arxiv.org/abs/2006.05535}{\textbf{Locally Private Graph Neural Networks}}\\
	\textit{ACM Conference on Computer and Communications Security (CCS 2021)}, Nov 2021

	\item{} Seyed Ali Osia, Ali Shahin Shamsabadi, \textbf{Sina Sajadmanesh}, \textit{et al.}\\
	\href{https://arxiv.org/abs/1703.02952}{\textbf{A Hybrid Deep Learning Architecture for Privacy-Preserving Mobile Analytics}}\\
	\textit{IEEE Internet of Things Journal}, May 2020
	
	\item{} \textbf{Sina Sajadmanesh}, Sogol Bazargani, Jiawei Zhang, and Hamid R. Rabiee\\
	\href{https://arxiv.org/abs/1710.00818}{\textbf{Continuous-Time Relationship Prediction in Dynamic Heterogeneous Information Networks}}\\
	\textit{ACM Transactions on Knowledge Discovery from Data}, Aug 2019
	
	\if\longversion1
	\item{} \textbf{Sina Sajadmanesh}, Jiawei Zhang, and Hamid R. Rabiee\\
	\href{https://arxiv.org/abs/1706.06783}{\textbf{NPGLM: A Non-Parametric Method for Temporal Link Prediction}}\\
	\textit{Technical Report, ArXiv e-prints}, Jun 2017
	\fi

	\item{} \textbf{Sina Sajadmanesh}, Sina Jafarzadeh, Seyed Ali Ossia, \textit{et al.}\\
	\href{https://arxiv.org/pdf/1610.08469}{\textbf{Kissing Cuisines: Exploring Worldwide Culinary Habits on the Web}}\\
	International World Wide Web Conference (WWW 2017) Companion, Apr 2017
	
	\item{} \textbf{Sina Sajadmanesh}, Hamid R. Rabiee and Ali Khodadadi\\
	\href{https://arxiv.org/pdf/1607.08821}{\textbf{Predicting Anchor Links between Heterogeneous Social Networks}}\\
	\textit{IEEE/ACM International Conference on Advances in Social Networks Analysis and Mining},  Aug 2016
	
\end{bibenum}


\section{Media Coverage}
	\textbf{MIT Technology Review}, {\href{https://www.technologyreview.com/s/602790/how-data-mining-reveals-the-worlds-healthiest-cuisines/}{How Data Mining Reveals the World’s Healthiest Cuisines}}, {3 Nov 2016}
	
	\textbf{The Independent},  {\href{https://www.indy100.com/article/healthy-diverse-top-healthiest-countries-cuisine-food-in-the-world-list-7412171}{These are the world's most diverse cuisines}}, {11 Nov 2016}
	
	\textbf{France 24},  {\href{https://www.france24.com/fr/20161115-algorithme-compare-cuisines-monde-matiere-dingredients-dapports-nutritionnels}{Un algorithme compare les cuisines du monde en matière d'ingrédients et ...}}, {15 Nov 2016}
	
%	\textbf{ReachMD},  {\href{https://reachmd.com/news/if-you-are-what-you-eat-regional-cuisines-have-a-major-impact-on-health/1306703/}{If you are what you eat: regional cuisines have a major impact on health}}, {4 Nov 2016}
	
	\textbf{Sciences et Avenir},  {\href{https://www.sciencesetavenir.fr/high-tech/data/diversite-nutrition-les-cuisines-du-monde-analysees-par-les-big-data_108012}{Les cuisines du monde passées au crible des big data}}, {14 Nov 2016}


\section{Talks and \\Presentations}

\textbf{Deep Learning on Graphs with Differential Privacy}
\begin{innerlist}
	\item[] Imperial-X, Imperial College London, March 2023
\end{innerlist}

\halfblankline

\textbf{Privacy-Preserving Machine Learning on Graphs}
\begin{innerlist}
	\item[] Socially Responsible AI Course, University of Illinois at Chicago (Remote), October 2022
\end{innerlist}

\halfblankline


\textbf{GAP: Differentially Private Graph Neural Networks with Aggregation Perturbation}
\begin{innerlist}
	\item[] L3S Research Center (Remote), Aug 2022
\end{innerlist}

\halfblankline

\textbf{Locally Private Graph Neural Networks}
\begin{innerlist}
	\item[] Graph Neural Networks User Group Meetup (Remote), Jul 2021
	\item[] AI4Media Workshop on Explainability, Robustness and Privacy in AI (Remote), Jun 2021
	\item[] Twitter Machine Learning Seminar (Remote), Jan 2021 
\end{innerlist}

\halfblankline

\textbf{Privacy-Preserving Deep Learning Over Graphs}
\begin{innerlist}
	\item[] \href{https://www.imperial.ac.uk/information-processing-and-communications-lab}{Information Processing and Communications Lab}, \href{https://www.imperial.ac.uk}{\textbf{Imperial College London}} (Remote), {Dec 2020}
\end{innerlist}




\section{Professional Services}

%\textbf{Journal Reviewer}
	\textbf{Organizer:}
	\begin{innerlist}
		\item[] \href{https://priv-fair-ai-uk.github.io}{Privacy and Fairness in AI for Health Workshop} (2023)
	\end{innerlist}
	\textbf{PC Member:}
	\begin{innerlist}
		\item[] \href{https://wisec2023.surrey.ac.uk/wiseml2023/}{ACM Workshop on Wireless Security and Machine Learning (WiseML)} (2023),
		\href{https://pair2struct-workshop.github.io/}{ICLR PAIR2Struct Workshop} (2022),
	\end{innerlist}
	\textbf{Reviewer:}
	\begin{innerlist}
		\item[] \href{http://aistats.org/aistats2023/}{International Conference on Artificial Intelligence and Statistics (AISTATS)} (2023),
		\href{https://logconference.org/}{Learning on Graphs Conference} (2022),
		\href{https://www.journals.elsevier.com/artificial-intelligence}{Artificial Intelligence Journal} (2022),
		\href{https://ieeexplore.ieee.org/xpl/RecentIssue.jsp?punumber=6687317}{IEEE Transactions on Big Data} (2021),
		\href{https://dp-ml.github.io/2021-workshop-ICLR/}{ICLR Workshop on Distributed and Private Machine Learning} (2021),
		\href{https://dl.acm.org/journal/tist}{ACM Transactions on Intelligent Systems and Technology} (2020),
		\href{https://www.springer.com/journal/13278}{Social Network Analysis and Mining Journal} (2020),
		\href{https://www.springer.com/journal/11280}{World Wide Web Journal} (2018)
	\end{innerlist}

\section{Honors\\and Awards}
\begin{innerlist}

\item
\textbf{Travel Grant}, 
{for attending CISPA Summer School on Trustworthy AI}, 
{Saarbrücken, Germany},
{2022}

\item
\textbf{Finalist}, 
{in CSAW Applied Research Competition for the best paper award in computer security}, 
{2021}

\item
\textbf{PhD research assistantship}, 
{Computer Science, University of Illinois at Urbana-Champaign}, 
{2018} (declined)

\item
\textbf{PhD studentship}
{Computer Science, Hong-Kong University of Science and Technology},
{2017} (declined)

% \item
% \textbf{Ranked 3rd}
% {in cum. GPA among undergraduate software engineering students, class of 2009},
% {University of Isfahan},
% {2014}

\item
\textbf{Ranked 6th}
{in nationwide university entrance exam for graduate studies in Artificial Intelligence},
{Iran},
{2014}

% \item
% \textbf{Ranked 15th}
% {in Iranian nationwide university entrance exam for graduate studies, field
% 	of Computer Networks and Security, among more than 30000 students},
% {2014}

% \item
% \textbf{Ranked 28th}
% {in 18th National Computer Olympiad for University Students at Tarbiat Modares University}
% , {Tehran, Iran}, 
% {2013}

\item
\textbf{Ranked 16th}
{in ACM-ICPC regional programming contest, Asia region},
{University of Tehran, Iran},
{2011}

\item
\textbf{Ranked 2nd}
{in nationwide collegiate programming contest},
{University of Kashan, Iran},
{2010}

\item
\textbf{Ranked among top 0.02\%}
{in Iran's nationwide university entrance exam for undergraduate studies},
{2009}
\end{innerlist}

% \if\longversion1
% \section{Memberships}

% {ACM Professional Member}, 2020 -- 2021

% {ACM Student Member}, 2011 -- 2014

% {ACM-ICPC Student Chapter}, \href{http://ui.ac.ir/EN}{{University of Isfahan}}, Esfahan, Iran, {2010 -- 2012}
% \fi
    


%\newpage
%\makeheading{Sina Sajadmanesh} 

% \if\longversion1
\section{Technical Skills} 

\textbf
{Programming Languages:}\\
{Python (Expert), Java (Moderate), C++ (Moderate)}

\halfblankline

\textbf
{Machine Learning \& Data Science:}\\
{PyTorch, PyTorch-Geometric, PyTorch-Lightning, Tensorflow, Scikit-Learn, Pandas}

\halfblankline

\textbf
{MLOps:}\\
{Weights \& Biases, Dask, Git}

\halfblankline

\textbf
{Privacy-Enhancing Technologies:}\\
{Flower, Opacus, Auto-DP}


% \fi

\if\longversion1
\section{References}
\textbf{Prof. Daniel Gatica-Perez}, Idiap Research Institute, EPFL

\enskip\begin{tabular}{p{6cm}l}
	Website: \href{https://idiap.ch/~gatica}{https://idiap.ch/\string~gatica} &
	Email: \href{mailto:daniel.gatica-perez@epfl.ch}{daniel.gatica-perez@epfl.ch}\\
\end{tabular}

\halfblankline

% \textbf{Prof. Hamid R. Rabiee}, Sharif University of Technology

% \enskip\begin{tabular}{p{6cm}l}
% 	Website: \href{http://sharif.ir/~rabiee}{http://sharif.ir/\string~rabiee} &
% 	Email: \href{mailto:rabiee@sharif.edu}{rabiee@sharif.edu}\\
% \end{tabular} 

% \halfblankline

\textbf{Prof. Hamed Haddadi}, Imperial College London

\enskip\begin{tabular}{p{6cm}l}
	Website: \href{https://haddadi.github.io/}{https://haddadi.github.io/} &
	Email: \href{mailto:h.haddadi@imperial.ac.uk}{h.haddadi@imperial.ac.uk}\\
\end{tabular}

\halfblankline

\textbf{Prof. Emiliano De Cristofaro}, University College London

\enskip\begin{tabular}{p{6cm}l}
	Website: \href{https://emilianodc.com/}{https://emilianodc.com/} &
	Email: \href{mailto:e.decristofaro@ucl.ac.uk}{e.decristofaro@ucl.ac.uk}\\
\end{tabular}
\fi

\end{document}

%%%%%%%%%%%%%%%%%%%%%%%%%% End CV Document %%%%%%%%%%%%%%%%%%%%%%%%%%%%%

%----------------------------------------------------------------------%
% The following is copyright and licensing information for
% redistribution of this LaTeX source code; it also includes a liability
% statement. If this source code is not being redistributed to others,
% it may be omitted. It has no effect on the function of the above code.
%----------------------------------------------------------------------%
% Copyright (c) 2007, 2008, 2009, 2010, 2011 by Theodore P. Pavlic
%
% Unless otherwise expressly stated, this work is licensed under the
% Creative Commons Attribution-Noncommercial 3.0 United States License. To
% view a copy of this license, visit
% http://creativecommons.org/licenses/by-nc/3.0/us/ or send a letter to
% Creative Commons, 171 Second Street, Suite 300, San Francisco,
% California, 94105, USA.
%
% THE SOFTWARE IS PROVIDED "AS IS", WITHOUT WARRANTY OF ANY KIND, EXPRESS
% OR IMPLIED, INCLUDING BUT NOT LIMITED TO THE WARRANTIES OF
% MERCHANTABILITY, FITNESS FOR A PARTICULAR PURPOSE AND NONINFRINGEMENT.
% IN NO EVENT SHALL THE AUTHORS OR COPYRIGHT HOLDERS BE LIABLE FOR ANY
% CLAIM, DAMAGES OR OTHER LIABILITY, WHETHER IN AN ACTION OF CONTRACT,
% TORT OR OTHERWISE, ARISING FROM, OUT OF OR IN CONNECTION WITH THE
% SOFTWARE OR THE USE OR OTHER DEALINGS IN THE SOFTWARE.
%----------------------------------------------------------------------%
