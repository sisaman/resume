\documentclass[11pt]{article}
\usepackage{resume}


\begin{document}

\name{Sina Sajadmanesh}
\website{https://sajadmanesh.com}
\email{sina.sajadmanesh@gmail.com}
\phone{(+41) 27-721-77-58}
\location{Lausanne, Switzerland}
\maketitle

\section{Career Summary}

\begin{outerlist}
	\item \textbf{Machine Learning Researcher}, \textbf{Software Engineer}, \textbf{Data Scientist}
	\begin{innerlist}
		\item \textbf{A self-motivated and diversely skilled professional} with 8 years of experience in designing, developing, and implementing machine learning models, AI solutions, and data-driven insights.
    \item \textbf{Adept in various machine learning frameworks, programming languages, and data analysis tools.} Passionate about leveraging cutting-edge technology to solve complex real-world problems and drive business growth.
    \item \textbf{A proven track record of success in research and development}, with several publications in top-tier conferences and journals. Demonstrated ability to collaborate effectively with cross-functional teams and lead projects from concept to completion.
    \item \textbf{A continuous learner with a growth mindset}, eager to contribute to the development of next-generation AI solutions and software that align with the strategic objectives and vision of a leading technology organization.
	\end{innerlist}
\end{outerlist}


\section{Skills and Expertise}

\begin{outerlist}
	\item \textbf{Programming and Scripting Languages:}\\
	{Python (Expert), Java (Moderate), C++ (Moderate), Shell Scripting (Moderate), SQL (Moderate), LaTeX (Moderate)}

	\item \textbf{Machine Learning \& Data Analysis:}\\
	{PyTorch, PyTorch-Geometric, Tensorflow, Tensorflow-Lite, Scikit-Learn, Pandas, Numpy, Jupyter, Matplotlib}

	\item \textbf{MLOps \& Related Technologies:}\\
	{Weights \& Biases, PyTorch-Lightning, Dask, Git, Linux, Docker}

	\item \textbf{Privacy-Enhancing Technologies:}\\
	{Flower (Federated Learning), Opacus (Differentially Private ML), Auto-DP (Privacy Accounting)}

  \item \textbf{Completed Courses:}\\
  Data Structures, Algorithms, Probability and Statistics, Artificial Intelligence, Information Retrieval, Software Engineering, Databases, Operating Systems, Computer Networks, Machine Learning, Complex Dynamical Networks, Performance Modeling of Computer Systems, Data and Network Security, Database Security and Privacy, Artificial Neural Networks (Deep Reinforcement Learning), Deep Learning for Natural Language Processing, Advanced Topics in Machine Learning, Introduction to Applied Data Science
\end{outerlist}


%%%% EDUCATION %%%%
\section{Education}
\begin{outerlist}

	\item Ph.D. in Electrical Engineering (GPA: 5.75 / 6)\\
	\href{https://www.epfl.ch/en/home/}{\textbf{École Polytechnique Fédérale de Lausanne (EPFL)}}, Lausanne, Switzerland \hfill {May 2019 -- Aug 2023 \textcolor{gray}{(Anticipated)}}

	\item M.Sc. in Information Technology Engineering (GPA: 18.1 / 20)\\
	\href{http://www.en.sharif.edu/}{\textbf{Sharif University of Technology}}, Tehran, Iran \hfill {Sep 2014 -- Sep 2016}

	\item B.Sc. in Computer Software Engineering (GPA: 16.19 / 20)\\
	\href{http://ui.ac.ir/EN}{\textbf{University of Isfahan}}, Esfahan, Iran \hfill {Sep 2009 -- Feb 2014}

\end{outerlist}


\section{Professional Experience}

\begin{outerlist}

	\item {Research Assistant}\\
	%   \href{https://www.idiap.ch/en/scientific-research/social-computing/index_html}{Social Computing Group},
	\textbf{\href{https://idiap.ch}{Idiap Research Institute}}, Martigny, Switzerland \hfill {May 2019 -- Present}\\
  \textit{Recruited as a doctoral student to contribute to \href{https://www.ai4media.eu/}{AI4Media}, a European research and innovation program focusing on ethical and trustworthy AI aiming to serve media and society. Developing privacy-preserving graph neural network (GNN) models using differential privacy to reduce privacy risks in real applications, such as recommendation systems and knowledge graphs.}
	\begin{innerlist}
		\item Conducted high-quality and impactful research: \textbf{published two papers} in top computer security/privacy conferences, namely \href{https://www.sigsac.org/ccs/CCS2021/}{ACM CCS} and \href{https://www.usenix.org/conference/usenixsecurity23/}{USENIX Security}, and one under review, \textbf{received over 60 citations} since 2021 and delivered \textbf{more than 6 invited talks} at top institutions and companies, such as \href{https://ix.imperial.ac.uk/}{Imperial College London}, \href{https://cs.uic.edu/}{University of Illinois}, and \href{https://twitter.com/}{Twitter}.
		\item \textbf{Implemented locally private GNN models} using \href{https://www.pyg.org/}{PyTorch-Geometric} that enables online social networks to learn from their users' data without compromising their privacy. \textbf{Received over 38 stars and 13 forks} on \href{https://github.com/sisaman/LPGNN}{GitHub}.
		\item \textbf{Developed a differentially private GNN model} using \href{https://www.pyg.org/}{PyTorch-Geometric}, \href{https://opacus.ai/}{Opacus}, and \href{https://github.com/yuxiangw/autodp}{Auto-DP} that enables privacy-preserving training and inference of GNNs. \textbf{Received over 20 stars and 1 fork} on \href{https://github.com/sisaman/GAP}{GitHub}.
		\item \textbf{Selected as a finalist} in \href{https://www.csaw.io/research}{CSAW Applied Research Competition} for the best paper award in computer security in Europe.
		\item \textbf{Offered a travel grant} to attend \href{https://cispa.de/en/summer-school-2022}{CISPA Summer School 2022 on Trustworthy Artificial Intelligence} in Germany.
	\end{innerlist}

	\item {Visiting PhD Student}\\ 
	\textbf{\href{https://www.turing.ac.uk/}{The Alan Turing Institute}}, London, UK \hfill {Feb 2023 -- Apr 2023}\\
  \textit{Joined the \href{https://www.turing.ac.uk/research/research-programmes/artificial-intelligence-ai/safe-and-ethical}{Safe and Ethical AI} group as a visiting PhD student to work on privacy-preserving ML algorithms on graphs. Worked on trustworthy machine learning on graphs at the intersection of privacy and robustness.}
	\begin{innerlist}
		\item \textbf{Co-organized a workshop} on \href{https://private-fair-ai.github.io/}{Privacy and Fairness in AI for Health} with \textbf{11 invited speakers} from top research institutes and companies, such as {Oxford}, {Microsoft}, and {DeepMind}, and \textbf{over 60 attendees}.
	\end{innerlist}

	\item {Research Intern}\\
	\textbf{\href{https://brave.com/}{Brave Software}}, San Francisco, CA, USA (Remote) \hfill {Mar 2022 -- May 2022}\\
  \textit{Recruited as a member of the research team to contribute to privacy-preserving machine learning research, aiming to improve Brave Browser's ads and news recommendation systems.}
	\begin{innerlist}
		\item \textbf{Hand-picked to work on a research project} about federated reinforcement learning for privacy-preserving recommendation.
		\item \textbf{Developed an experimental framework} to evaluate the performance of federated neural bandits under client heterogeneity using \href{https://pytorch.org/}{PyTorch} and \href{https://flower.dev/}{Flower} for server-side simulation and \href{https://www.tensorflow.org/lite}{Tensorflow-Lite} for mobile clients.
	\end{innerlist}


	\item {Big-Data Engineer} \\
	\href{http://ictic.sharif.ir}{\textbf{Sharif ICT Innovation Center}}, Tehran, Iran \hfill {Sep 2018 -- May 2019}\\
  \textit{Served in a part-time role as a member of an R\&D team to develop a native big-data processing platform for the center.}
	\begin{innerlist}
		\item \textbf{Conducted a comprehensive study on massively scalable graph databases}, such as \href{https://neo4j.com/}{Neo4j}, OrientDB, and JanusGraph, by setting up, configuring, and benchmarking them on a computing cluster.
		\item \textbf{Developed a dashboard} to analyze and discover key insights from customer's data using \href{https://www.elastic.co/kibana}{Kibana}, \href{https://www.elastic.co/elasticsearch/}{Elasticsearch}, and \href{https://cassandra.apache.org/}{Cassandra}.
	\end{innerlist}


	\item {Research Assistant}\\
	%   \textcolor{darkblue}{Data Science and Machine Learning Lab}, 
	\href{http://www.en.sharif.edu/}{\textbf{Sharif University of Technology}}, Tehran, Iran \hfill {Nov 2014 -- May 2019}\\
  \textit{Joined the {Data Science and Machine Learning Lab} as a master's student and continued as a research assistant after graduation. Worked on a range of research projects in the areas of privacy-preserving machine learning, web data science, and social and information network analysis.}

	\begin{innerlist}
    \item Conducted high-quality and impactful research: \textbf{published four papers} in top-tier venues, including \href{https://ieee-iotj.org/}{IEEE IoTJ}, \href{https://dl.acm.org/journal/tkdd}{ACM TKDD}, and \href{https://thewebconf.org/www2017/}{TheWebConf}, \textbf{with over 350 citations} since 2016.
		\item Worked on a hybrid mobile-server learning architecture based on Siamese fine-tuning and split learning to make non-private pre-trained deep learning models privacy-preserving at the inference stage.
		\item \textbf{Conducted large-scale data analysis} using a collection of recipes published on the web and their content, aiming to understand cuisines and culinary habits around the world. \textbf{Received media coverage} from prominent news outlets, such as \href{https://www.technologyreview.com/s/602790/how-data-mining-reveals-the-worlds-healthiest-cuisines/}{MIT Technology Review}, \href{https://www.indy100.com/article/healthy-diverse-top-healthiest-countries-cuisine-food-in-the-world-list-7412171}{The Independent}, and \href{https://www.france24.com/fr/20161115-algorithme-compare-cuisines-monde-matiere-dingredients-dapports-nutritionnels}{France 24}.
		\item \textbf{Developed time-aware link prediction algorithms} over heterogeneous social networks using recurrent neural networks and non-parametric machine learning.
		\item Received \textbf{research assistantship} from
    \href{https://www.c3sr.com/}{IBM-ILLINOIS Center for Cognitive Computing Systems Research} (declined due to visa issues).
	\end{innerlist}

\end{outerlist}

\section{Professional Services}

\begin{outerlist}
	\item \textbf{Organizer}\\
	\href{https://priv-fair-ai-uk.github.io}{Privacy and Fairness in AI for Health Workshop} (2023)

	\item \textbf{Program Committee Member}\\
	\href{https://wisec2023.surrey.ac.uk/wiseml2023/}{ACM Workshop on Wireless Security and Machine Learning (WiseML)} (2023),
	\href{https://pair2struct-workshop.github.io/}{ICLR PAIR2Struct Workshop} (2022),

	\item \textbf{Reviewer}\\
	\href{http://aistats.org/aistats2023/}{International Conference on Artificial Intelligence and Statistics (AISTATS)} (2023),
	\href{https://logconference.org/}{Learning on Graphs Conference} (2022),
	\href{https://www.journals.elsevier.com/artificial-intelligence}{Artificial Intelligence Journal} (2022),
	\href{https://ieeexplore.ieee.org/xpl/RecentIssue.jsp?punumber=6687317}{IEEE Transactions on Big Data} (2021),
	\href{https://dp-ml.github.io/2021-workshop-ICLR/}{ICLR Workshop on Distributed and Private Machine Learning} (2021),
	\href{https://dl.acm.org/journal/tist}{ACM Transactions on Intelligent Systems and Technology} (2020),
	\href{https://www.springer.com/journal/13278}{Social Network Analysis and Mining Journal} (2020),
	\href{https://www.springer.com/journal/11280}{World Wide Web Journal} (2018)

\end{outerlist}

\section{Publications}

\begin{outerlist}
	\item {Sina Sajadmanesh} and Daniel Gatica-Perez\\
	\href{https://arxiv.org/abs/2304.08928}{{ProGAP: Progressive Graph Neural Networks with Differential Privacy Guarantees}}\\
	\textit{Technical Report, ArXiv e-prints}, Apr 2023

	\item {Sina Sajadmanesh}, Ali Shahin Shamsabadi, Aurélien Bellet, and Daniel Gatica-Perez\\
	\href{https://arxiv.org/abs/2203.00949}{{GAP: Differentially Private Graph Neural Networks with Aggregation Perturbation}}\\
	\textit{USENIX Security Symposium (USENIX Security 23)}, Aug 2023

	\item {Sina Sajadmanesh} and Daniel Gatica-Perez\\
	\href{https://arxiv.org/abs/2006.05535}{{Locally Private Graph Neural Networks}}\\
	\textit{ACM Conference on Computer and Communications Security (CCS 2021)}, Nov 2021

	\item Seyed Ali Osia, Ali Shahin Shamsabadi, {Sina Sajadmanesh}, \textit{et al.}\\
	\href{https://arxiv.org/abs/1703.02952}{{A Hybrid Deep Learning Architecture for Privacy-Preserving Mobile Analytics}}\\
	\textit{IEEE Internet of Things Journal}, May 2020

	\item {Sina Sajadmanesh}, Sogol Bazargani, Jiawei Zhang, and Hamid R. Rabiee\\
	\href{https://arxiv.org/abs/1710.00818}{{Continuous-Time Relationship Prediction in Dynamic Heterogeneous Information Networks}}\\
	\textit{ACM Transactions on Knowledge Discovery from Data}, Aug 2019

	\item {Sina Sajadmanesh}, Jiawei Zhang, and Hamid R. Rabiee\\
	\href{https://arxiv.org/abs/1706.06783}{{NPGLM: A Non-Parametric Method for Temporal Link Prediction}}\\
	\textit{Technical Report, ArXiv e-prints}, Jun 2017

	\item {Sina Sajadmanesh}, Sina Jafarzadeh, Seyed Ali Ossia, \textit{et al.}\\
	\href{https://arxiv.org/pdf/1610.08469}{{Kissing Cuisines: Exploring Worldwide Culinary Habits on the Web}}\\
	International World Wide Web Conference (WWW 2017) Companion, Apr 2017

	\item {Sina Sajadmanesh}, Hamid R. Rabiee and Ali Khodadadi\\
	\href{https://arxiv.org/pdf/1607.08821}{{Predicting Anchor Links between Heterogeneous Social Networks}}\\
	\textit{IEEE/ACM International Conference on Advances in Social Networks Analysis and Mining},  Aug 2016

\end{outerlist}


% \section{Media Coverage}
% \begin{outerlist}[itemsep=0pt]
% 	\item \textbf{MIT Technology Review}, {\href{https://www.technologyreview.com/s/602790/how-data-mining-reveals-the-worlds-healthiest-cuisines/}{How Data Mining Reveals the World’s Healthiest Cuisines}} \hfill {Nov 2016}

% 	\item \textbf{The Independent},  {\href{https://www.indy100.com/article/healthy-diverse-top-healthiest-countries-cuisine-food-in-the-world-list-7412171}{These are the world's most diverse cuisines}} \hfill {Nov 2016}

% 	\item \textbf{France 24},  {\href{https://www.france24.com/fr/20161115-algorithme-compare-cuisines-monde-matiere-dingredients-dapports-nutritionnels}{Un algorithme compare les cuisines du monde en matière d'ingrédients et d'apports nutritionnels}} \hfill {Nov 2016}

% 	\item \textbf{Sciences et Avenir},  {\href{https://www.sciencesetavenir.fr/high-tech/data/diversite-nutrition-les-cuisines-du-monde-analysees-par-les-big-data_108012}{Les cuisines du monde passées au crible des big data}} \hfill {Nov 2016}
% \end{outerlist}


% \section{Honors and Awards}

% \begin{outerlist}[itemsep=0pt]

% 	\item
% 	\textbf{Travel Grant},
% 	{for attending CISPA Summer School on Trustworthy AI},
% 	{Saarbrücken, Germany} \hfill
% 	{2022}

% 	\item
% 	\textbf{Finalist},
% 	{in CSAW Applied Research Competition for the best paper award in computer security} \hfill
% 	{2021}

% 	\item
% 	\textbf{Graduate research assistantship},
% 	{IBM-ILLINOIS Center for Cognitive Computing Systems Research (declined)} \hfill
% 	{2018}

% 	\item
% 	\textbf{PhD studentship} in
% 		{Computer Science, Hong-Kong University of Science and Technology (declined)
% 		} \hfill
% 	{2017}
% 	% \item
% 	% \textbf{Ranked 3rd}
% 	% {in cum. GPA among undergraduate software engineering students, class of 2009},
% 	% {University of Isfahan},
% 	% {2014}

% 	\item
% 	\textbf{Ranked 6th}
% 	{in nationwide university entrance exam for graduate studies in Artificial Intelligence},
% 	{Iran} \hfill
% 	{2014}

% 	% \item
% 	% \textbf{Ranked 15th}
% 	% {in Iranian nationwide university entrance exam for graduate studies, field
% 	% 	of Computer Networks and Security, among more than 30000 students},
% 	% {2014}

% 	% \item
% 	% \textbf{Ranked 28th}
% 	% {in 18th National Computer Olympiad for University Students at Tarbiat Modares University}
% 	% , {Tehran, Iran}, 
% 	% {2013}

% 	% \item
% 	% \textbf{Ranked 16th}
% 	% {in ACM-ICPC regional programming contest, Asia region},
% 	% {University of Tehran, Iran},
% 	% {2011}

% 	\item
% 	\textbf{Ranked 2nd}
% 	{in nationwide collegiate programming contest},
% 	{University of Kashan, Iran} \hfill
% 	{2010}

% 	\item
% 	\textbf{Ranked among top 0.02\%}
% 	{in Iran's nationwide university entrance exam for undergraduate studies} \hfill
% 	{2009}

% \end{outerlist}

% \if\longversion1
% \section{Memberships}

% {ACM Professional Member}, 2020 -- 2021

% {ACM Student Member}, 2011 -- 2014

% {ACM-ICPC Student Chapter}, \href{http://ui.ac.ir/EN}{{University of Isfahan}}, Esfahan, Iran, {2010 -- 2012}
% \fi




\end{document}
